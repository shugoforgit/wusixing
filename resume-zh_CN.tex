% !TEX TS-program = xelatex
% !TEX encoding = UTF-8 Unicode
% !Mode:: "TeX:UTF-8"

\documentclass{resume}
\usepackage{zh_CN-Adobefonts_external} % Simplified Chinese Support using external fonts (./fonts/zh_CN-Adobe/)
%\usepackage{zh_CN-Adobefonts_internal} % Simplified Chinese Support using system fonts
\usepackage{linespacing_fix} % disable extra space before next section
\usepackage{cite}

\begin{document}
\pagenumbering{gobble} % suppress displaying page number

\name{吴思兴}

\basicInfo{
  \email{shugoforgml@gmail.com} \textperiodcentered\ 
  \phone{(+86) 180-6057-0063} \textperiodcentered\ 
  \linkedin[]}
 
\section{\faGraduationCap  教育背景}
\datedsubsection{\textbf{闽江学院}, 福州}{2017 -- 2021}
\textit{学士}\ 软件工程

\section{\faUsers\ 实习/项目经历}
\datedsubsection{\textbf{福州掌中云科技有限公司} 福州}{2021年6月 -- 2021年8月}
\role{php后端开发}
\begin{onehalfspacing}
\begin{itemize}
  \item 负责微信小程序阅读平台的后端API设计与开发,保障接口的高可用性与安全性
  \item 使用PHP进行业务逻辑开发,主要模块包括用户登录、书籍信息管理、阅读进度同步及支付接口对接
  \item 参与前端部分页面的开发与数据渲染,实现前后端高效协作
\end{itemize}
\end{onehalfspacing}

\datedsubsection{\textbf{福州热游网络科技有限公司}}{2021年8月 -- 2025年1月}
\role{中台、游戏服务端}
\begin{onehalfspacing}
\begin{itemize}
  \item 负责游戏中台部门技术规划与团队协作,主导游戏账服系统、数据中台及运维工具链的建设
  \item 使用PHP重构并维护高可用游戏账服系统,稳定服务多款线上游戏
  \item 负责基于Golang和PHP的数据统计平台与运维支持系统的功能迭代与性能维护
  \item 与两个游戏项目后端的开发与迭代,并独立开发前端通用联想输入组件,提升了用户操作效率
  \item 设计实现基于OAuth2的统一认证鉴权体系,解决了各平台认证方式不一的痛点,提升了内部系统安全性
\end{itemize}
\end{onehalfspacing}

\datedsubsection{\textbf{上海共识引擎网络科技有限公司}}{2025年1月 -- 至今}
\role{后端开发}
\begin{onehalfspacing}
优雅的 \LaTeX\ 简历模板, https://github.com/billryan/resume
\begin{itemize}
  \item 容易定制和扩展
  \item 完善的 Unicode 字体支持,使用 \XeLaTeX\ 编译
  \item 支持 FontAwesome 4.5.0
\end{itemize}
\end{onehalfspacing}

% Reference Test
%\datedsubsection{\textbf{Paper Title\cite{zaharia2012resilient}}}{May. 2015}
%An xxx optimized for xxx\cite{verma2015large}
%\begin{itemize}
%  \item main contribution
%\end{itemize}

\section{\faCogs\ IT 技能}
% increase linespacing [parsep=0.5ex]
\begin{itemize}[parsep=0.5ex]
  \item 编程语言: golang == php > solidity == python == javascript... 
  \item 平台: linux、windows
  \item 云平台: 腾讯云、阿里云
\end{itemize}

\section{\faInfo\ 其他}
% increase linespacing [parsep=0.5ex]
\begin{itemize}[parsep=0.5ex]
  \item GitHub: https://github.com/shugoforgit
\end{itemize}

%% Reference
%\newpage
%\bibliographystyle{IEEETran}
%\bibliography{mycite}
\end{document}
